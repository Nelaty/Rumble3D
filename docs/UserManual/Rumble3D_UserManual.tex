\documentclass[12p, paper=a4, leqno, colorinlistoftodos]{article}
\usepackage{wrapfig}
\usepackage[utf8]{inputenc}
\usepackage[ngerman]{babel}
\usepackage{amsmath}
\usepackage{amsfonts} 
\usepackage{amssymb}
\usepackage{enumerate}
\usepackage{graphicx}
\usepackage{wrapfig}

\usepackage[parfill]{parskip}
\usepackage{subcaption}
\usepackage{fix-cm}
\usepackage{transparent}
\usepackage{color}
\usepackage[table]{xcolor}
\usepackage{tocloft, url}
\usepackage{hyperref}
\usepackage{etoolbox}
\usepackage{fancyhdr}
\usepackage[german]{fancyref}
\usepackage{lipsum}	
\usepackage{sectsty}
\usepackage{pdfpages}
\usepackage{tcolorbox}
\usepackage{todonotes}


\usepackage{enumitem}
\usepackage[bottom]{footmisc}

\usepackage{forest}
\usepackage{float}
\usepackage{longtable}
\usepackage[normalem]{ulem}
\usepackage{multirow}
\usepackage[font=small,labelfont=bf,tableposition=top]{caption}
\usepackage{tikz}
\usepackage{calc}
\usepackage{geometry}
\usepackage{listings,lstautogobble}
\usepackage{algorithm}
\usepackage{algorithmicx}
\usepackage[noend]{algpseudocode}

\usepackage{array}
\usepackage{booktabs}
\usepackage{xcolor,colortbl}
\usepackage{setspace}
\usepackage{titlesec}
\usepackage[nottoc,notlot,notlof,numbib]{tocbibind}

\usepackage{cite}


\hypersetup{
	colorlinks,
	linkcolor={black},
	citecolor={blue!50!black},
	urlcolor={blue!80!black}	
}

% BEGIN COLORS
\definecolor{CBlue}{RGB}{1,0,119}
\definecolor{CBlack}{rgb}{0.0,0.0,0.0}
\definecolor{CLightGrey}{rgb}{0.664,0.664,0.664}
\definecolor{LightOrange}{rgb}{1.0,0.9,0.63}
\definecolor{DarkOrange}{rgb}{1.0,0.8,0.22}
\definecolor{LightCyan}{rgb}{1.0,0.9,0.63}
\definecolor{CWhite}{RGB}{0,0,0}
\definecolor{CLink}{rgb}{0.5,0.0,0.0}

\newcommand{\CTitle}{CBlue}
% BEGIN Table row/column colors{102,217,239}
\definecolor{CEven0}{RGB}{208,238,244}
\definecolor{CUneven0}{RGB}{255,255,255}
\definecolor{CEven1}{rgb}{1.0,1.0,1.0}
\definecolor{CUneven1}{rgb}{1.0,1.0,1.0}
% END Table row/column colors
% END COLORS


% BEGIN NEWENVIRONMENTS
\newenvironment{packed_itemize}
{\begin{itemize}
		\setlength{\itemsep}{0pt}
		\setlength{\parskip}{0pt}
		\setlength{\parsep}{0pt}
	}{\end{itemize}}

\newenvironment{packed_description}
{\begin{description}
		\setlength{\itemsep}{0pt}
		\setlength{\parskip}{0pt}
		\setlength{\parsep}{0pt}}
	{\end{description}}

\newenvironment{packed_enumerate}
{\begin{enumerate}
		\setlength{\itemsep}{0pt}
		\setlength{\parskip}{0pt}
		\setlength{\parsep}{0pt}
	}{\end{enumerate}}

% END NEWENVIRONMENTS


% BEGIN NEWCOMMANDS
% tables
\newcommand{\tabularPaddingSmall}{\renewcommand{\arraystretch}{0.75}}
\newcommand{\tabularPaddingDefault}{\renewcommand{\arraystretch}{1}}
\newcommand{\tabularPaddinglarge}{\renewcommand{\arraystretch}{1.25}}
\newcommand{\tabularPaddingLarge}{\renewcommand{\arraystretch}{1.5}}
\newcommand{\tabularPaddingLARGE}{\renewcommand{\arraystretch}{2}}
\newcommand{\tabularPaddingBot}{\vspace{0.5cm}}

\newcommand{\tabitem}{~~\llap{\textbullet}~~}
\newcommand{\coloredHline}[1]{\arrayrulecolor{#1}\hline}
\newcommand{\minitab}[2][l]{\begin{tabular}{#1}#2\end{tabular}}

% quoting
\newcommand{\quoteSource}[2]{\glqq{#1}\grqq, \textit{#2}}
\newcommand{\quoteQuick}[1]{\glqq{#1}\grqq}
\newcommand{\quoteAuthor}[2]{\quoteQuick{#1}, von \textit{#2}}
\newcommand{\quoteCite}[2]{\quoteQuick{\textit{#1}}#2}
\newcommand{\srcSelf}{\\\text{Quelle: Eigene Darstellung}}
\newcommand{\citePage}[2]{\cite{#1} Seite: {#2}}
\newcommand{\citeEquation}[1]{\text{\hspace{1cm}{#1}}}

% paragraphs
\newcommand{\parTiny}{\vspace{2mm}\\}
\newcommand{\parSmall}{\\~\par}
\newcommand{\parNoIndent}{\\\noindent}
\newcommand{\parWrapfig}{\\\text{ }\\}

% math


% 
\newcommand{\emphFold}[1]{\textbf{#1}}

% page layout
\newcommand{\geometryNormal}{\newgeometry{top = 2cm, left=2.5cm,bottom=2cm, right = 4cm}}
\newcommand{\geometryToc}{\newgeometry{top = 3cm, left=3cm,bottom=3cm, right = 3cm}}

% todo notes
\newcommand{\todoLowp}[1]{\todo[color=green!40]{#1}}
\newcommand{\todoMedp}[1]{\todo[color=orange!40]{#1}}
\newcommand{\todoHighp}[1]{\todo[color=red!40]{#1}}
\newcommand{\todoFig}[1]{\missingfigure{#1}}

% titles
\colorlet{ctcolorchapternum}{black}

\newcommand\mysectionformatCore[1]{%
	\vspace{-3em}\parbox[b]{\dimexpr\textwidth-1em\relax}{\raggedright#1}}
\newcommand{\titleformatCore}
{
	\titleformat
	{\section}[display]%
	{\color{black}\Large\bfseries}%
	{\vspace{-18em}\raggedleft{%
			{\color{ctcolorchapternum}\fontsize{60}{60}{\selectfont\thesection}}%
		}\hspace{1em}%
	}%
	{0pt}%
	{\mysectionformatCore}%
	[{\vspace{-0.5em}%
		\rule{0.75\textwidth}{1pt}%
		\rule[-3pt]{0.25\textwidth}{4pt}}]
	
	\titleformat
	{\subsection}[block]%
	{\color{black}\large\bfseries}%
	{\thesubsection\text{ }}%
	{0pt}{}[]%
	
	\titleformat
	{\subsubsection}[block]%
	{\color{black}\normalsize\bfseries}%
	{\thesubsubsection\text{ }}%
	{0pt}{}[]%
	
	\titleformat{\paragraph}
	{\normalfont\normalsize\bfseries}{\theparagraph}{1em}{}
	\titlespacing*{\paragraph}
	{0pt}{3.25ex plus 1ex minus .2ex}{1.5ex plus .2ex}
}

\newcommand\mysectionformatAppendix[1]{#1}
\newcommand{\titleformatAppendix}
{
	\titleformat
	{\section}[display]%
	{\color{black}\Large\bfseries}%
	{}{0pt}{\mysectionformatAppendix}[]
}

% END NEWCOMMANDS

% BEGIN RENEW COMMANDS
% END RENEW COMMANDS

\newlistof{links}{lks}{List of Links}
\newcommand\externallink[1]{%
	\refstepcounter{links}%
	\footnote{\url{#1}}%
	\addcontentsline{lks}{links}{%
		\protect\numberline{\thelinks}%
		\protect\url{#1}}%
}
\newcommand\externallinkIL[1]{%
	\refstepcounter{links}%
	\addcontentsline{lks}{links}{%
		\protect\numberline{\thelinks}%
		\protect\url{#1}}%
}

\newcolumntype{L}[1]{>{\raggedleft\let\newline\\\arraybackslash\hspace{0pt}}m{#1}}
\newcolumntype{R}[1]{>{\raggedright\let\newline\\\arraybackslash\hspace{0pt}}m{#1}}
%\newcolumntype{C}[1]{>{\raggedcenter\let\newline\\\arraybackslash\hspace{0pt}}m{#1}}

\newcolumntype{C}[1]{>{\raggedcenter\hspace{0pt}}p{#1}}


\author{Matthias Elmar Gensheimer}
\title{Rumble3D Manual}
\date{\today}


% BEGIN HEADER
\newcommand{\headerNormal}
{
	\pagestyle{fancy}
	\fancyhf{}
	\lhead{\rightmark}
	\rhead{\thepage}
}

\newcommand{\headerEmpty}
{
	\fancyhf{}
	\renewcommand{\headrulewidth}{0pt}
	\lhead{}
	\rhead{}
}

\newcommand{\headerErklaerung}
{
	\pagestyle{fancy}
	\fancyhf{}
	\rhead{\thepage}
}
% END HEADER

% BEGIN FOOTER
\usepackage{scrlayer}
\DeclareNewLayer[
foreground,
foot,
hoffset=0pt,
width=\paperwidth,
contents={\parbox{\layerwidth}{\centering}}
]{PageMarkCentredToPage}
\RedeclarePageStyleByLayers{plain}{PageMarkCentredToPage}
% END FOOTER

% BEGIN TOC SETUP
\makeatletter
\patchcmd{\l@section}
{\hfil}
{\leaders\hbox{\normalfont$\m@th\mkern \@dotsep mu\hbox{.}\mkern \@dotsep mu$}\hfill}
{}{}
\makeatother
% END TOC SETUP

% BEGIN PAGE SETUP
\geometryNormal
\setlength{\extrarowheight}{3pt}
% END PAGE SETUP


\titleformatCore
\headerNormal

\begin{document}
	\pagenumbering{Roman} 
	
	\lstset{language=C++,
		keywordstyle=\color{RoyalBlue},
		basicstyle=\ttfamily,
		commentstyle=\ttfamily\itshape\color{gray},
		stringstyle=\ttfamily,
		showstringspaces=false,
		breaklines=true,
		frameround=ffff,
		rulecolor=\color{black},
		autogobble=true
	}
	
	\begin{titlepage}
		\begin{flushright}
			\includegraphics[width=0.45\textwidth]{resources/Evelin_Logo_Final_4fbg}
			\includegraphics[width=0.45\textwidth]{resources/hs_kempten_logo}
		\end{flushright}
		\begin{center}
			\topskip0pt
			\vspace*{\fill}
			\LARGE\textbf{Rumble3D User Manual}\\		
			
			%\LARGE{}\\
			\vspace{1cm}
			%\Large{EVELIN Project}\\
			\vspace{2cm}
			\includegraphics[width=0.30\textwidth]{resources/Rumble3DLogo}
			
			\vfill
			\Large{\textcopyright 2018 Matthias Gensheimer}\\
			\Large{All Rights Reserved.}
		\end{center}
		
	\end{titlepage}
	
	% Add an empty page
	\mbox{}
	\thispagestyle{empty}
	\newpage
	
	\pagebreak
	
	\geometryNormal
	\tableofcontents
	\geometryNormal
	\setstretch{1.30}
	
	\pagenumbering{arabic} 
	\section{Introduction}

	
	\pagebreak
	\section{First steps}

	
	\pagebreak
	\section{Collision detection}
	One major part of this physics engine is the detection and resolution of collisions between rigid bodies. First off all the collision detection system will be explained.
		
		% TODO: collision detection pipeline visualization + INPUT/OUTPUT
		
		\subsection{Broad phase}
		The broad phase takes in a number of rigid bodies and outputs a number of \textbf{CollisionPairs}. The task of the broad phase is to eliminate collision pairs, which do not collide. The generated \textbf{CollisionPairs} are only potential contacts, which means that there can be false positives. There should never be false negatives (e.q. eliminated collision pairs, that actually collide).
		
		
		\subsection{Intermediate phase}
		An intermediate phase takes in a number of \textbf{CollisionPairs} and outputs a (probably) smaller number of \textbf{CollisionPairs}. An intermediate phase behaves just like a broad phase, with the only difference being that multiple intermediate phases can be concatenated.
		
		\subsection{Narrow phase}
		A narrow phase takes a number of \textbf{CollisionPairs} as input and outputs a number of contacts. These contacts are then used in the collision resolution later on.
		
			\subsubsection{Collision Algorithms}
			\begin{table}[ht]
				\begin{tabular}{|l || l | l | l |}
					\hline
					& \textbf{Box} & \textbf{Sphere} & \textbf{Plane}\\
					\hline
					\hline
					\textbf{Box} & Box-Box & Box-Sphere & Box-Plane\\
					\hline
					\textbf{Sphere} & Sphere-Box & Sphere-Sphere & Sphere-Plane\\
					\hline
					\textbf{Plane}& Plane-Box & Plane-Sphere & Plane-Plane\\
					\hline
				\end{tabular}
			\end{table}
	
	\pagebreak
	\section{Collision resolution}
	
	
	
	
	
	
	\pagenumbering{alph}
	\newpage
	\appendix
	\listoffigures
	\listoftables
	
	% TODO: glossary
	
\end{document}


